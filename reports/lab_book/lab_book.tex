\documentclass[10pt,a4paper]{report}
 \usepackage[utf8]{inputenc}
 \usepackage{amsmath}
 \usepackage{amsfonts}
 \usepackage{amssymb}
 \author{Cyril Matthey-Doret}
 \begin{document}
 \title{\textbf{Master Project}\\ Lab book}
 \maketitle
 \chapter{Introduction}
Samples were single-end sequenced using a ddRAD protocole and digested with ecoRI and mseI. There are 2 separate libraries with two different illumina adaptors (6 and 12). 

\begin{center}
\begin{tabular}{l| c c}
 \multicolumn{1}{r}{Data summary:} \\
 \hline
library & lib7 & lib7b \\
raw reads & 163,506,603 & 133,574,055 \\
containing adaptors & 23.25\% & 45.84\% \\
fragment size & 302bp & 302bp \\
mean quality score & 34.88 & 35.05 \\
$>=$ Q30 bases & 92.62\% & 92.13\% \\
\end{tabular}
\end{center}
 \chapter{Processing reads}
 RAD-seq data was split into 2 separate libraries: 7 and 7b. Together, the libraries contain 173 F4 individuals from 12 different F3 mothers. 
 \section{Quality control}
 fastqc was used for quality control, separately on each file, and on all files together in the library.
 \section{Demultiplexing}
 The process-radtags program from stacks was used for demultiplexing and removal of Illumina adaptors. The operation was performed separately for libraries 7 and 7b:

\noindent \textbf{process\_radtags -p raw/ -o processed/ -b /barcodes -e ecoRI --filter\_illumina -E phred33 -r -c -q --adapter\_1 adapter --adapter\_mm 2}

\begin{center}
\begin{tabular}{c c}
lib7 & Truseq adapter, index 6 \footnote{GATCGGAAGAGCACACGTCTGAACTCCAGTCACGCCAATATCTCGTATGCCGTCTTCTGCTTG}\\ 
lib7b & TruSeq adapter, index 12 \footnote{GATCGGAAGAGCACACGTCTGAACTCCAGTCACCTTGTAATCTCGTATGCCGTCTTCTGCTTG}\\
\end{tabular}
\end{center}
\section{Trimming adaptors}
This step is performed by process radtags at the same time as demultiplexing.
\chapter{STACKS pipeline}
\end{document} 
